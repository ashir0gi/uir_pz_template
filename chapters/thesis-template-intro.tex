\chapter*{Введение}
\label{sec:afterwords}
\addcontentsline{toc}{chapter}{Введение}

\textbf{Актуальность.}


	В академической среде всё шире применяется лямбда‑исчисление как фундаментальная модель вычислений. Ручная обработка β‑редукции и сложных термовых преобразований задача не из простых. Это порождает потребность в инструментах, которые оптимизируют этот процесс и предоставляют визуализацию своей работы.


	\emph{Краткая история:}
        \begin{itemize}
          \item 1932, Чёрч — зарождение λ‑исчисления как формализма функций и рекурсии;
          \item 1964, Ландин — SECD‑машина для исполнения λ‑выражений с управлением стеком и окружением;
          \item 1975, Плоткин — классификация стратегий «call‑by‑name» и «call‑by‑value»;
          \item 2007, Кривин — абстрактная машина для call‑by‑name с замыканиями.
        \end{itemize}


	\emph{Существующие проблемы:}  
        \begin{itemize}
          \item Недостаточная модульность традиционных визуализаторов — каждая новая стратегия или синтаксическое расширение требует правки ядра.  
          \item Ограниченная кроссплатформенность и устаревшие реализации, не пригодные для современных образовательных сред.
        \end{itemize}

\textbf{Новизна работы.}  
    \begin{itemize}
      \item Предложена концепция «расширяемой предструктуры» для представления λ‑термов, позволяющая подключать новые атомики, редукции и синтаксические конструкции декларативно, без изменения движка.  
      \item Реализована кроссплатформенная (.NET 8.0+, F\#) библиотека, сочетающая настраиваемый парсер на FParsec и абстрактную машину бета‑редукции с пошаговым и полным режимами.
    \end{itemize}

\textbf{Оригинальная суть исследования.}  
    \begin{itemize}
      \item Формализация модели аппликативной среды на основе расширенной предструктуры, где термы и атомики описываются через интерфейсы \texttt{ITff} и \texttt{IAtomic}, а расширение системы сводится к регистрации новых компонентов.  
      \item Интеграция комбинаторного подхода к синтаксическому анализу с механизмом приоритетных операторов и колбэков для гибкой настройки лексики и парсинга λ‑выражений.  
      \item Построение абстрактной машины, способной демонстрировать каждый шаг β‑редукции и одновременно обеспечивать рекурсивную нормализацию вектором вызовов \texttt{evaluateOneStep}.
    \end{itemize}

\textbf{Содержание работы.}
    \begin{itemize}
      \item \textbf{Глава 1} посвящена обзору литературы: классификация алгоритмов бета‑редукции, сравнительный анализ AST‑ и графовых подходов, а также обзор классических абстрактных машин.  
      \item \textbf{Глава 2} описывает теоретические основы: модель расширяемой предструктуры, формализация редукций, структуры данных для λ‑термов и выбор синтаксиса учебного языка.  
      \item \textbf{Глава 3} раскрывает проектирование API: конфигурация парсера (\texttt{ILanguageParser}, \texttt{Config}), интерфейсы машины (\texttt{IAbstractMachine}, \texttt{Code}, \texttt{State}) и архитектуру подключения пользовательских расширений.  
      \item \textbf{Глава 4} демонстрирует реализацию и экспериментальную проверку: структура NuGet‑пакета, ключевые фрагменты кода, примеры NUnit‑тестов и сравнительный анализ с существующими инструментами.
    \end{itemize}

