\chapter*{Заключение}
\addcontentsline{toc}{chapter}{Заключение}

В ходе выполнения работы получены следующие результаты:

\begin{itemize}
    \item \textbf{Аналитические результаты:}
    \begin{itemize}
        \item Проведён сравнительный анализ стратегий бета-редукции (нормальный, аппликативный порядки, call-by-need), выявлены их преимущества и ограничения для различных вычислительных сценариев.
        \item Исследованы существующие реализации абстрактных машин, определены их архитектурные особенности и области применения.
        \item Проанализированы возможности платформы .NET и языка F\# для реализации функциональных и асинхронных вычислений, подтверждена их пригодность для поставленной задачи.
    \end{itemize}

    \item \textbf{Теоретические результаты:}
    \begin{itemize}
        \item Разработана формальная модель аппликативной вычислительной среды с расширяемой предструктурой, обеспечивающая модульность и масштабируемость системы.
        \item Формализованы правила подстановки и преобразования термов с учётом конфлюэнтности бета-редукции, гарантирующие корректность вычислений.
        \item Определены принципы построения синтаксиса учебного языка, сочетающие минимализм, расширяемость и семантическую прозрачность.
    \end{itemize}

    \item \textbf{Инженерные результаты:}
    \begin{itemize}
        \item Спроектирована модульная архитектура системы с чётким разделением компонентов (парсер, абстрактная машина, атомики) через интерфейсы.
        \item Разработан API для синтаксического анализатора, поддерживающий динамическую конфигурацию операторов, ключевых слов и правил разбора.
        \item Создана модель абстрактной машины пошаговой редукции, обеспечивающая как полное вычисление, так и пошаговое выполнение.
    \end{itemize}

    \item \textbf{Практические результаты:}
    \begin{itemize}
        \item Реализован синтаксический анализатор на основе библиотеки FParsec, поддерживающий настраиваемый синтаксис и обработку ошибок.
        \item Создан Бета-редуктор, выполняющий пошаговую и полную редукцию лямбда-термов с сохранением промежуточных состояний.
        \item Разработан набор модульных тестов, подтверждающих корректность работы парсера и редуктора на различных примерах.
    \end{itemize}
\end{itemize}

\textbf{Перспективные направления для дальнейшей работы включают:}
\begin{itemize}
	\item Развитие системы визуализации шагов редукции с поддержкой интерактивного управления и возможностью экспорта результатов в форматы XAML и JSON для интеграции с внешними инструментами.
	\item Добавление возможности компиляции промежуточных состояний машины в Microsoft Intermediate Language (MsIL) для последующего анализа и отладки на уровне .NET CLR.\end{itemize}
\end{itemize}

%%% Local Variables:
%%% TeX-engine: xetex
%%% eval: (setq-local TeX-master (concat "../" (seq-find (-cut string-match ".*-3-pz\.tex$" <>) (directory-files ".."))))
%%% End:
