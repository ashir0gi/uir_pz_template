\chapter{Анализ проблематики программной реализации Бета‑Редукции}
\label{chapter1}

Лямбда‑исчисление было предложено А. Чёрчем в 1932 году как формализм для исследования функций и рекурсии \cite{ChurchRosser}. Основные объекты в лямбда‑исчислении, называемые \emph{лямбда‑термами}, строятся по трём правилам: введение \emph{переменной}, \emph{абстракции} и \emph{аппликации} — например, если $x$ и $y$ — переменные, то $\lambda x.\,x\,y$ является термом .  

Операция \emph{бета‑редукции} формализует применение функции к аргументу. Формально, для терма вида $(\lambda x.\,M)\,N$ результатом одного шага бета‑редукции является терм $M[x := N]$, где $M[x := N]$ обозначает подстановку $N$ вместо всех свободных вхождений $x$ в $M$ .  

Ключевым свойством бета‑редукции является \emph{конфлюэнтность} (теорема Чёрча — Россера): если терм $M$ может быть приведён двумя разными способами к $N_1$ и $N_2$, то существует терм $P$, в который $N_1$ и $N_2$ могут быть дополнительно редуцированы — это гарантирует однозначность нормальной формы независимо от выбранной стратегии редукции .  
\section{Теоретические основы Бета‑Редукции}
\label{sec:theory-beta}

% Основным теоретическим фундаментом аналитической части в этой работе стал научный труд  
% В.\,Э.~Вольфенгагена \cite{Wolfengagen2004}. Вольфенгаген даёт всестороннюю классификацию \emph{аппликативных вычислительных систем}, включая:
% \begin{itemize}
%   \item формализацию \emph{термов первого порядка} и их расширение до \emph{термов высшего порядка},  
%   \item построение абстрактных машин через чисто синтаксические правила редукции — вплоть до граф‑редукции,  
%   \item подробное описание \emph{стратегий управления контекстом} (environment, stack) при бета‑редукции,  
%   \item анализ производительности различных моделей редукции на формальных примерах, включая сравнение памяти и числа шагов редукции.
% \end{itemize}

% Таким образом, идеи из \cite{Wolfengagen2004} легли в основу:
% \begin{enumerate}
%   \item выбора представления термов в виде \emph{атомиков} и \emph{композитов},  
%   \item реализации \emph{внутреннего буфера окружения} (environment buffer) для эффективной подстановки,  
% \end{enumerate}


\subsection{Определение лямбда‑термов}
Лямбда‑термы строятся по трём правилам (см. \cite{Barendregt1984}):
\begin{enumerate}
  \item Любая переменная \(x\) является термом.
  \item Если \(M\) и \(N\) — термы, то их аппликация \((M\ N)\) также является термом.
  \item Если \(x\) — переменная, а \(M\) — терм, то \(\lambda x.\,M\) (лямбда‑абстракция) является термом.
\end{enumerate}
Обозначим множество всех лямбда‑термов как \(\Lambda\).  

\subsection{Свободные и связанные переменные; альфа‑конверсия}
В терме \(\lambda x.\,M\) все вхождения \(x\) в \(M\) считаются \emph{связанными}, остальные переменные называются \emph{свободными}. Обозначим множества свободных и связанных переменных терма \(M\) как \(\mathrm{FV}(M)\) и \(\mathrm{BV}(M)\) соответственно \cite{HindleySeldin2008}.  

Для предотвращения конфликтов при подстановке вводят операцию \emph{альфа‑конверсии}: 
\[
  \lambda x.\,M \;\equiv_\alpha\;\lambda y.\,(M[x\mapsto y]),
\]
где \(y\) не входит в \(\mathrm{FV}(M)\). Альфа‑конверсия сохраняет семантику терма, меняя только связанные имена переменных.

\subsection{Определение бета‑редукции}
Бета‑редукция — это операция применения функции к аргументу:
\[
  (\lambda x.\,M)\;N \;\to_\beta\; M[x := N],
\]
где \(M[x := N]\) обозначает \emph{подстановку} \(N\) вместо свободных вхождений \(x\) в \(M\) \cite{Pierce2002}.  

Подстановка \(M[x:=N]\) определяется рекурсивно:
\[
  \begin{aligned}
    x[x:=N] &= N,\\
    y[x:=N] &= y,\quad y\neq x,\\
    (P\ Q)[x:=N] &= (P[x:=N])\;(Q[x:=N]),\\
    (\lambda y.\,P)[x:=N] &=
      \begin{cases}
        \lambda y.\,P[x:=N], & y\notin\mathrm{FV}(N),\\
        \lambda z.\,(P[y:=z])[x:=N], & y\in\mathrm{FV}(N),\; z\notin\mathrm{FV}(P)\cup\mathrm{FV}(N).
      \end{cases}
  \end{aligned}
\]
Второй случай лямбда‑абстракции требует переименования (альфа‑конверсии), если \(y\) встречается в \(N\), чтобы избежать захвата свободных переменных.

\subsection{Нормальная форма и стратегия редукции}
Терм \(M\) называется \emph{бета‑нормальным}, если в нём нет подтермов вида \((\lambda x.\,P)\,Q\). Нормальная форма терма, если она существует, единственна (до альфа‑конверсии) благодаря конфлюэнтности бета‑редукции \cite{ChurchRosser}.  

Различают две классические стратегии применения шагов бета‑редукции \cite{Plotkin1975}:
\begin{itemize}
  \item \textbf{Нормальный порядок} (normal order): в каждый момент редуцируется самый внешний левый редекс. Эта стратегия гарантирует достижение нормальной формы, если та существует.
  \item \textbf{Аппликативный порядок} (applicative order): сначала редуцируются аргументы, затем само применение. Может не завершиться, даже если нормальная форма существует.
\end{itemize}


\subsection{Свойства бета‑редукции}
\paragraph{Конфлюэнтность.} Теорема Чёрча – Россера утверждает, что если \(M \to_\beta^* N_1\) и \(M \to_\beta^* N_2\), то существует \(P\) такое, что \(N_1 \to_\beta^* P\) и \(N_2 \to_\beta^* P\) \cite{ChurchRosser}. Это гарантирует, что нормальная форма (если существует) единственна.  

\paragraph{Строгая нормализация.} В нетипизированном лямбда‑исчислении существуют термы, не имеющие нормальной формы (например, \(\Omega = (\lambda x.\,x\,x)\,(\lambda x.\,x\,x)\)). В типизированных системах (например, системах просто типизированного лямбда‑исчисления) достигается свойство сильной нормализации: все термы приводятся к нормальной форме за конечное число шагов \cite{Barendregt1984}.



\section{Анализ особенностей Бета‑Редукций}
\label{sec:features-beta}


\subsection{AST vs.\ графовое представление термов}
Традиционный подход к бета‑редукции основан на работе с абстрактным синтаксическим деревом (AST): каждый шаг редукции создаёт новый AST, в котором подтермы копируются при подстановке — это просто в реализации, но порождает значительные издержки по памяти и времени на копирование одинаковых поддеревьев \cite{Pierce2002}\cite{HindleySeldin2008}.  

Альтернативно, в \emph{графовой редукции} (graph reduction) используют ориентированный ациклический граф (DAG), где общий подтерм хранится единожды и на него указывают несколько ссылок. При редукции заменяются лишь указатели, что позволяет избежать многократного копирования и реализовать мемоизацию (sharing) для ленивых стратегий — основа реализации ленивого Haskell в GHC на механизме STG \cite{PeytonJones1992}\cite{Barendregt1993}.  

\subsection{Модели управления окружением и стеком}
Классическая \textsc{SECD}‑машина Ландина разделяет состояние на четыре регистра: стек (S), окружение (E), управляющую последовательность (C) и дамп (D). При приведении аппликации элементы попарно обрабатываются с сохранением контекста в дампе — подход наглядный, но сравнительно громоздкий в реализации \cite{Landin1964}\cite{Felleisen1987}.  

\textsc{CEK}‑машина упрощает SECD: объединяет дамп и стек продолжений в одну структуру Kontinuation (K), сохраняя при этом разделение Control (C) и Environment (E). Она особенно хорошо подходит для call‑by‑value семантики и теоретических доказательств корректности вычислений \cite{Felleisen1987}\cite{Plotkin1975}.  

Машина Кривина оптимизирована для call‑by‑name: она хранит окружение в виде ассоциаций переменных с замыканиями и использует стек аргументов без дополнительного дампинга, что даёт минимальные накладные расходы при редукции \cite{Krivine2007}\cite{Barendregt1984}.  

\subsection{Оптимизации и стратегии редукции}
\emph{Call‑by‑need} (ленивая) стратегия сочетает normal order с мемоизацией: первый раз при обращении к редексу происходит редукция и результат сохраняется, при последующих обращениях переиспользуется — это устраняет повторные вычисления, сохраняя гарантию достижения нормальной формы \cite{Barendregt1993}\cite{Wadler1987}.  

Для строгих языков (call‑by‑value) разработаны оптимизации вроде \emph{tail‑recursion elimination} и \emph{deforestation}, которые устраняют промежуточные структуры при последовательных аппликациях, что критично для производительности в реализованных на C и JVM интерпретаторах \cite{PeytonJones1992}\cite{Wadler1990}.  


\section{Сравнительный анализ алгоритмов Бета‑Редукций}
\label{sec:comparison-beta}


\begin{table}[h]
\caption{Сравнительный анализ стратегий бета‑редукции}
\label{tbl:cmp-strategies}
\centering
\footnotesize
\begin{tabular}{|c|l|c|c|c|}
\hline
№ & Стратегия & Порядок & НФ & Эффективность \\
\hline
1 & Normal & Слева наружу & Да, если есть & Низкая \\
2 & CBN & Наружу, без замены & Не гарантирована & Высокая \\
3 & CBV & Слева внутрь & Быстрая, но не всегда & Высокая \\
4 & Lazy & С мемоизацией & Как Normal & Наилучшая \\
5 & Full & Любой редекс & Да, теоретически & Непрактична \\
\hline
\end{tabular}
\end{table}

\medskip

\noindent
Из таблицы \ref{tbl:cmp-strategies} видно, что наивные AST‑алгоритмы просты в реализации, но имеют большие накладные расходы по памяти и не всегда гарантируют достижение нормальной формы (для applicative order). Абстрактные машины CEK и Krivine оптимизированы для своих стратегий: первая эффективна для strict‑языков, вторая — для ленивых. Graph‑редукция сочетает гарантии нормальной формы (call‑by‑need) с малым расходом памяти благодаря разделению поддеревьев.





\section{Сравнительный анализ программных средств для Бета‑Редукции}
\label{sec:tools-beta}


В процессе подготовки к проекту было проведено исследование программных решений, содержащих реализацию бета‑редукции. Основу анализа составили интерпретаторы, proof assistant’ы и экспериментальные среды, в которых реализована одна или несколько стратегий редукции. В ходе исследования учитывались открытые исходные коды, спецификации и научные публикации, подтверждающие архитектурные особенности инструментов.  

На основе результатов анализа была составлена сравнительная таблица. Особое внимание уделено выбору языков реализации, стратегии редукции, сложности расширения и целевого применения инструментов.

\begin{table}[h]
\caption{Сравнение программных средств для бета‑редукции}
\label{tbl:cmp-tools-beta}
\centering
\small
\begin{tabular}{|c|l|l|l|l|}
\hline
№ & Инструмент & Язык & Стратегия & Особенности \\
\hline
1 & Mace & F\# (.NET) & Normal, Full & Модульная архитектура, AST \\
2 & GHCi & Haskell & Call‑by‑need & STG‑машина, граф‑редукция \\
3 & Coq & OCaml & Call‑by‑value & Формализация λ-исчисления \\
4 & Agda & Haskell & Normal/CBV & Зависимая типизация \\
5 & PAKCS & Java & CBV/CBN & Функц.-логическая семантика \\
6 & LC Playground & JS/Web & Normal/App & Визуализация, браузерный \\
\hline
\end{tabular}
\end{table}


\medskip

\noindent
Выбор программной платформы для продолжения работы был обусловлен не только результатами анализа, но и техническим контекстом проекта. Репозиторий, в рамках которого велась предыдущая разработка, использовал язык F\# и платформу .NET. Этот выбор оказался обоснованным:

\begin{itemize}
    \item F\# предоставляет выразительные средства для работы с абстрактными синтаксическими деревьями и рекурсивными структурами данных, что критично при реализации $\beta$‑редукции.
    \item Типовая система F\# позволяет надёжно моделировать лямбда‑термы и окружения.
    \item Язык ориентирован на функциональное программирование, что приближает его семантику к λ‑исчислению.
\end{itemize}

Таким образом, несмотря на наличие производительных и развитых решений на Haskell и OCaml, язык F\# был выбран в качестве основной платформы для продолжения работы по нескольким причинам:
\begin{enumerate}
    \item Совместимость с исходным кодом и существующей архитектурой.
    \item Высокая читаемость кода и простота создания прототипов.
    \item Возможность эффективной интеграции с другими модулями на .NET.
\end{enumerate}



\section{Выводы}

В ходе работы были проведены теоретический и практический анализ стратегий бета-редукции в лямбда-исчислении, исследованы особенности их реализации, а также выполнен сравнительный анализ существующих подходов. На основе полученных результатов сделаны следующие выводы:

\begin{enumerate}
    \item Выполнен теоретический анализ стратегий бета-редукции (Normal Order, Call-by-Name, Call-by-Value, Lazy/Call-by-Need и Full Beta) с точки зрения корректности, завершимости и эффективности. Выявлено, что стратегия Normal Order обеспечивает достижение нормальной формы, если таковая существует, и может быть полезна при разработке систем верификации и символьных вычислений.
    
    \item Проведено исследование реальных реализаций стратегий редукции, включая работы из академических публикаций \cite{GarciaPerez2019, Krivine2007Abstract}, а также примеры из классических источников, таких как книга Вольфенгагена \cite{Wolfengagen2004}. Это позволило уточнить особенности практического применения стратегий в различных контекстах.

    \item Сравнительный анализ стратегий (см. таблицу~\ref{tbl:cmp-strategies}) показал, что при всех теоретических преимуществах нормального порядка, в прикладных задачах часто предпочтительнее использовать ленивую стратегию (Call-by-Need), обеспечивающую лучшую производительность за счёт мемоизации. Однако с точки зрения теоретической чистоты и универсальности, стратегия Normal Order остаётся обоснованным выбором.

    \item В связи с тем, что код проекта, в который была интегрирована абстрактная машина, уже был реализован на языке F\#, принято решение продолжить работу в этом же технологическом стеке. Это позволило сократить затраты на интеграцию и использовать имеющуюся инфраструктуру.

\end{enumerate}

Результаты проведённого анализа легли в основу проектирования и реализации абстрактной машины бета-редукции и повлияли на выбор стратегий вычислений и организацию кода. Далее предполагается сосредоточиться на расширении функциональности машины и интеграции с другими компонентами проекта.



\section{Постановка задачи дипломной работы/курсового проекта}

Это всегда последний пункт. Далее пишется постановка задачи, на основе выданного задания. Это должен быть связный текст в объеме до 1-1,5 страниц. В этом разделе необходимо раскрыть цели и задачи УИРа/диплома. 

%%% Local Variables:
%%% TeX-engine: xetex
%%% eval: (setq-local TeX-master (concat "../" (seq-find (-cut string-match ".*-3-pz\.tex$" <>) (directory-files ".."))))
%%% End:
