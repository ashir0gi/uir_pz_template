\chapter*{Реферат}
\thispagestyle{plain}

Общий объем основного текста, без учета приложений — \pageref{end_of_main_text} страниц, с учетом приложений — 61. Количество использованных источников — 16. Количество приложений — 2.

\noindent \uppercase{лямбда-исчисление, бета‑редукция, абстрактная машина, F\#, FParsec}

Целью данной работы является разработка расширяемой библиотеки на платформе .NET/F\# для синтаксического анализа и пошаговой бета‑редукции лямбда‑термов с возможностью конфигурирования стратегий редукции и операторов.

В первой главе проводится обзор литературы по аппликативным методикам организации вычислений, сравнительный анализ стратегий бета‑редукции и возможностей платформы .NET для реализации асинхронных и параллельных вычислений.

Во второй главе описывается используемая модель расширяемой аппликативной вычислительной среды, выбор и формализация стратегий редукции, разработка структур данных и метаданных для представления термов, а также выбор синтаксиса учебного языка.

В третьей главе приводятся результаты проектирования: архитектура системы, API абстрактной машины бета‑редуктора и синтаксических анализаторов, описаны их интерфейсы и взаимодействие компонентов.

В четвертой главе изложена реализация и экспериментальная проверка: ключевые фрагменты кода парсера и машины, описание тестовых примеров, результаты тестирования и сравнение с существующими аналогами.

В приложении \ref{app-format} представлены основные структуры разработанного модуля.

В приложении \ref{app-format1} представлены тесты основного модуля.

